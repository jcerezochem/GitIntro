% Version 1: first release 50-60min
% Version 2: cut..


\documentclass[xcolor=dvipsnames,10pt]{beamer}
% \usepackage[english]{babel}
\usepackage[T1]{fontenc}
% \usepackage[sfdefault]{cabin}
% \usepackage[T1]{fontenc}
% \usepackage[LY1]{fontenc}
\usepackage{multicol}
\usepackage{multirow}
\usepackage{comment}
\usepackage{xfrac}
\usepackage{cancel}
\usepackage{graphicx}
\usepackage{epstopdf}
\usepackage{hyperref}
\usepackage{amsmath}
\usepackage{amssymb}

%Nice diagrams
\usepackage{tikz}
\usetikzlibrary{calc,trees,positioning,arrows,chains,shapes.geometric,%
    decorations.pathreplacing,decorations.pathmorphing,shapes,%
    matrix,shapes.symbols,%
    mindmap% Crear mapas como el de la intro
    }
\tikzset{
	%Define standard arrow tip
	>=stealth',
	%Define style for different line styles
	help lines/.style={dashed, thick},
	axis/.style={<->},
	important line/.style={thick},
	connection/.style={thick, dotted},
}

\hypersetup{
  linkcolor=black,
  urlcolor=blue,
  pdffitwindow=true,
  pdfnewwindow=true,
  linktoc=all,
%   hidelinks=false,
  colorlinks=true
}

% Opciones para "usetheme"
% pagetype: nombre de para las páginas en la parte inferior (p.ej. Página, Page)
% pageofpages: de/of/etc...
%
% logo: logotipo de la parte superior de la portada, y en la inferior de los frames.
%
% titleimg: imagen de la portada; proporciones 1410x438
\usetheme[pagetype=Page, pageofpages=of, logo=img/umu, titleimg=img/title] {UMU}
\usecolortheme{UMU}


% Para páginas off-topic, no incluidas en el número de páginas de la presentación
\newcommand{\beginbackup}{
   \newcounter{framenumbervorappendix}
   \setcounter{framenumbervorappendix}{\value{framenumber}}
}
\newcommand{\backupend}{
   \addtocounter{framenumbervorappendix}{-\value{framenumber}}
   \addtocounter{framenumber}{\value{framenumbervorappendix}} 
}
\newcommand{\textttb}[1]{{\color{blue!50!black} \texttt{#1}}}

% Fuentes: El tema original usa la Nexus Serif y la Nexus Sans, ambas de pago, por lo que 
%          recomiendo comentar/editar estas líneas y/o usar otras fuentes.
% \renewcommand{\rmdefault}{NexusSerifPro}
% \renewcommand{\sfdefault}{NexusSansPro}

\usepackage{xcolor}
\usepackage{color}
\usepackage{colortbl}

\usepackage{listings}
\lstdefinestyle{Output}{
%   language=bash,
%   firstline=2,% Supress the first line that begins with `%`
  escapechar=\%,
  basicstyle=\small\ttfamily,
  numbers=none,
  numberstyle=\tiny,
  numbersep=3pt,
  frame=tb,
  columns=fullflexible,
  backgroundcolor=\color{white},
  linewidth=0.9\linewidth,
  xleftmargin=0.1\linewidth
}

%%%%%%%%%%%%%%%%%%%%%%%%%%%%%%%%%%%%%%%%%%%%%%%%
%DEFINICIONES DE COLORES
% Texto en segundo plano
\definecolor{gris}{rgb}{0.5,0.5,0.5}
\definecolor{gris2}{rgb}{0.8,0.8,0.8}
\definecolor{gris_oscuro}{rgb}{0.3,0.3,0.3}
% Texto en primer plano (defecto)
\definecolor{negro}{rgb}{0.0,0.0,0.0}
% Chapucilla
\definecolor{blanco}{rgb}{1.0,1.0,1.0}
% 
\definecolor{verde}{rgb}{0.0,1.0,0.0}
%
\definecolor{verde2}{rgb}{0.0,0.5,0.0}
% 
\definecolor{rojo}{rgb}{1.0,0.0,0.0}
%
\definecolor{violeta}{rgb}{0.5,0.0,1.0}
%
\definecolor{naranja}{rgb}{1.0,0.5,0.0}
% 
\definecolor{azul}{rgb}{0.0,0.0,1.0}
% 
\definecolor{azul2}{rgb}{0.4,0.6,1.0}
%Más
\definecolor{BlueViolet}{rgb}{0.5,0.0,0.6}


\newenvironment{annot}{\begin{block}{}}{\end{block}}

% Mostrar esquema de contenidos antes de cada sección; descomentar para activar.
%\AtBeginSection[]{
%	\begin{frame}<beamer>
%		\frametitle{Contenidos}
%		
%		\small
%		\raggedcolumns
%		\begin{multicols}{2}
%			\tableofcontents[sectionstyle=show/shaded,subsectionstyle=show/show/shaded]
%		\end{multicols}
%	
%	\end{frame}
%}

% Opciones para dibujo con tikz
\tikzstyle{sqr} = [draw, thin, fill=blue!20, minimum height=1.6em]
\tikzstyle{elp} = [ellipse, draw, thin, fill=green!20, minimum height=1.6em]
\tikzstyle{rsqr} = [draw, thin, fill=red!20, minimum height=1.6em]
\tikzstyle{relp} = [ellipse, draw, thin, fill=red!20, minimum height=1.6em]

% Definir opciones: tema, título, fecha, etc
\title[Intro to git]{\textbf{Short introduction to git}}
% \subtitle{\vspace*{-0.2cm} An {\itshape express} introduction to GROMACS}
\author{Javier Cerezo}
\institute{}
\date{Javier Cerezo\\\vspace*{0.2cm}\today} %September 2017}
\tema{}

\setbeamertemplate{section in toc}[circle]
\setbeamertemplate{subsection in toc}[square]


%%%%%%%%%%%%%%%%%%%%%%%%%%%%%%%%%%%%%%%%%%%%%%%%%%%%%%%%%%%%%%%%%%%%%%%%%%%%%%%%%%%% 
%%%%%%%%%%%%%%%%%%%%%%%%%%%%%%%%%%%%%%%%%%%%%%%%%%%%%%%%%%%%%%%%%%%%%%%%%%%%%%%%%%%% 
%%%%%%%%%%%%%%%%%%%%%%%%%%%%%%%%%%%%%%%%%%%%%%%%%%%%%%%%%%%%%%%%%%%%%%%%%%%%%%%%%%%% 

\begin{document}

\newcommand{\wpl}{$\omega^{+}$}
\newcommand{\wm}{$\omega^{-}$}
\newcommand{\cis}{\textit{cis}}
\newcommand{\trans}{\textit{trans}}
\newcommand{\alltrans}{\textit{all-trans}}
\newcommand{\monocis}{\textit{monocis}}
\newcommand{\dicis}{\textit{dicis}}

\begin{frame}[t,plain]
  \titlepage
\end{frame}

% \section*{Contents}
% \begin{frame}[c]
%   \frametitle{Contents}
%   \small
%   \tableofcontents
%   
% \end{frame}

\section{Motivation}
\begin{frame}
 \frametitle{Why using git?}
 
 ``Old school'' version management might be, somehow, not so efficient:
 
 \begin{center}
 \includegraphics[width=6cm]{figures/why_git.png}  
 \end{center}

 
 If the above folder organisation is not unfamiliar to you and/or:
 
 \begin{itemize}
  \item you want to keep traceability of your work
  \item \textbf{you work with other people}
 \end{itemize}

\vspace*{0.3cm}
 
 \begin{center}
 \begin{minipage}{0.2\textwidth}
  \includegraphics[width=2cm]{figures/Git-Logo-2Color.png}  
 \end{minipage}
  \begin{minipage}{0.7\textwidth}
 can make your life (much) easier
  \end{minipage}
 \end{center}
  
\end{frame}

\section{Basics}
\begin{frame}
 \frametitle{What is git?}
 
 \begin{minipage}{1.0\textwidth}
  ``Git is a free and open source \textbf{distributed version control system} designed to handle everything from small to very large projects with speed and efficiency.''
 \end{minipage}
 \vspace*{0.4cm}
 
 \pause
 
 Let's go back to our working directory{\color{white}, git adds a new layer (actually 2)}
 
 \begin{center}
 \includegraphics[width=6.5cm]{figures/gitless_world.png}\\
 \includegraphics[width=6.5cm]{figures/gitless_world_simil.png}  
 \end{center}
 
\end{frame}
%%%%%%%%%%%%%%%%%%%%%%%%%%%%%%%
\addtocounter{framenumber}{-1}
%%%%%%%%%%%%%%%%%%%%%%%%%%%%%%%
\begin{frame}
 \frametitle{What is git?}
 
 \begin{minipage}{1.0\textwidth}
  ``Git is a free and open source \textbf{distributed version control system} designed to handle everything from small to very large projects with speed and efficiency.''
 \end{minipage}
 \vspace*{0.4cm}
 
 Let's go back to our working directory, git adds a new layer (actually 2) 
 
 \begin{center}
 \includegraphics[width=6.5cm]{figures/git_world.png}\\
 \includegraphics[width=6.5cm]{figures/gitless_world_simil.png}  
 \end{center}
 
\end{frame}
%%%%%%%%%%%%%%%%%%%%%%%%%%%%%%%
\addtocounter{framenumber}{-1}
%%%%%%%%%%%%%%%%%%%%%%%%%%%%%%%
\begin{frame}
 \frametitle{What is git?}
 
 \begin{minipage}{1.0\textwidth}
  ``Git is a free and open source \textbf{distributed version control system} designed to handle everything from small to very large projects with speed and efficiency.''
 \end{minipage}
 \vspace*{0.4cm}
 
 Let's go back to our working directory, git adds a new layer (actually 2)
 
 \begin{center}
 \includegraphics[width=6.5cm]{figures/git_world.png}\\
 \includegraphics[width=6.5cm]{figures/git_world_simil.png}  
 \end{center}
 
\end{frame}

\begin{frame}
 \frametitle{What is git?}
 
 Git is the set of tools that are used to manage such layers
  \vspace*{0.4cm}
 
 \begin{center}
 \includegraphics[width=5.5cm]{figures/git_3_states.png}
 \end{center}
 
\end{frame}
%%%%%%%%%%%%%%%%%%%%%%%%%%%%%%%
\addtocounter{framenumber}{-1}
%%%%%%%%%%%%%%%%%%%%%%%%%%%%%%%
\begin{frame}
 \frametitle{What is git?}
 
 Git is the set of tools that are used to manage such layers
  \vspace*{0.4cm}
 
 \begin{center}
 \includegraphics[width=5.5cm]{figures/git_3_states_commands.png}
 \end{center}
 
\end{frame}

%=============================================
%
% Slides with steps in git with Example
%
%=============================================
\begin{frame}
 \frametitle{What is git? Example}
 
 \begin{center}
  \includegraphics[width=7cm]{figures/git_step_0.png}
 \end{center}
 \vspace*{-0.5cm}
 
 \begin{itemize}
  \color{white}
  \item[] \texttt{vi test.dat}
  \item[] \texttt{git init}
  \item[] \texttt{git add test.dat}
  \item[] \texttt{git commit -m "Create my new file"}
  \item[] \texttt{rm test.dat}
  \item[] \texttt{git checkout test.dat}
  \item[] \texttt{vi test.dat}
  \item[] \texttt{git add test.dat}
  \item[] \texttt{git commit -m "Update my file"}
  \item[] \texttt{git checkout [1] test.dat}
 \end{itemize}


\end{frame}
%%%%%%%%%%%%%%%%%%%%%%%%%%%%%%%
\addtocounter{framenumber}{-1}
%%%%%%%%%%%%%%%%%%%%%%%%%%%%%%%
\begin{frame}
 \frametitle{What is git? Example}
 
 \begin{center}
  \includegraphics[width=7cm]{figures/git_step_1.png}
 \end{center}
 \vspace*{-0.5cm}
 
 \begin{itemize}
  \color{gray}
  \color{black}
  \item   \texttt{vi test.dat}
  \color{white}
  \item[] \texttt{git init}
  \item[] \texttt{git add test.dat}
  \item[] \texttt{git commit -m "Create my new file"}
  \item[] \texttt{rm test.dat}
  \item[] \texttt{git checkout test.dat}
  \item[] \texttt{vi test.dat}
  \item[] \texttt{git add test.dat}
  \item[] \texttt{git commit -m "Update my file"}
  \item[] \texttt{git checkout [1] test.dat}
 \end{itemize}


\end{frame}
%%%%%%%%%%%%%%%%%%%%%%%%%%%%%%%
\addtocounter{framenumber}{-1}
%%%%%%%%%%%%%%%%%%%%%%%%%%%%%%%
\begin{frame}
 \frametitle{What is git? Example}
 
 \begin{center}
  \includegraphics[width=7cm]{figures/git_step_init.png}
 \end{center}
 \vspace*{-0.5cm}
 
 \begin{itemize}
  \color{gray}
  \item[] \texttt{vi test.dat}
  \color{black}
  \item   \texttt{git init}
  \color{white}
  \item[] \texttt{git add test.dat}
  \item[] \texttt{git commit -m "Create my new file"}
  \item[] \texttt{rm test.dat}
  \item[] \texttt{git checkout test.dat}
  \item[] \texttt{vi test.dat}
  \item[] \texttt{git add test.dat}
  \item[] \texttt{git commit -m "Update my file"}
  \item[] \texttt{git checkout [1] test.dat}
 \end{itemize}


\end{frame}
%%%%%%%%%%%%%%%%%%%%%%%%%%%%%%%
\addtocounter{framenumber}{-1}
%%%%%%%%%%%%%%%%%%%%%%%%%%%%%%%
\begin{frame}
 \frametitle{What is git? Example}
 
 \begin{center}
  \includegraphics[width=7cm]{figures/git_step_add.png}
 \end{center}
 \vspace*{-0.5cm}
 
 \begin{itemize}
  \color{gray}
  \item[] \texttt{vi test.dat}
  \item[] \texttt{git init}
  \color{black}
  \item   \texttt{git add test.dat}
  \color{white}
  \item[] \texttt{git commit -m "Create my new file"}
  \item[] \texttt{rm test.dat}
  \item[] \texttt{git checkout test.dat}
  \item[] \texttt{vi test.dat}
  \item[] \texttt{git add test.dat}
  \item[] \texttt{git commit -m "Update my file"}
  \item[] \texttt{git checkout [1] test.dat}
 \end{itemize}


\end{frame}
%%%%%%%%%%%%%%%%%%%%%%%%%%%%%%%
\addtocounter{framenumber}{-1}
%%%%%%%%%%%%%%%%%%%%%%%%%%%%%%%
\begin{frame}
 \frametitle{What is git? Example}
 
 \begin{center}
  \includegraphics[width=7cm]{figures/git_step_commit.png}
 \end{center}
 \vspace*{-0.5cm}
 
 \begin{itemize}
  \color{gray}
  \item[] \texttt{vi test.dat}
  \item[] \texttt{git init}
  \item[] \texttt{git add test.dat}
  \color{black}
  \item   \texttt{git commit -m "Create my new file"}
  \color{white}
  \item[] \texttt{rm test.dat}
  \item[] \texttt{git checkout test.dat}
  \item[] \texttt{vi test.dat}
  \item[] \texttt{git add test.dat}
  \item[] \texttt{git commit -m "Update my file"}
  \item[] \texttt{git checkout [1] test.dat}
 \end{itemize}


\end{frame}
%%%%%%%%%%%%%%%%%%%%%%%%%%%%%%%
\addtocounter{framenumber}{-1}
%%%%%%%%%%%%%%%%%%%%%%%%%%%%%%%
\begin{frame}
 \frametitle{What is git? Example}
 
 \begin{center}
  \includegraphics[width=7cm]{figures/git_step_delwd.png}
 \end{center}
 \vspace*{-0.5cm}
 
 \begin{itemize}
  \color{gray}
  \item[] \texttt{vi test.dat}
  \item[] \texttt{git init}
  \item[] \texttt{git add test.dat}
  \item[] \texttt{git commit -m "Create my new file"}
  \color{black}
  \item   \texttt{rm test.dat}
  \color{white}
  \item[] \texttt{git checkout test.dat}
  \item[] \texttt{vi test.dat}
  \item[] \texttt{git add test.dat}
  \item[] \texttt{git commit -m "Update my file"}
  \item[] \texttt{git checkout [1] test.dat}
 \end{itemize}

\end{frame}
%%%%%%%%%%%%%%%%%%%%%%%%%%%%%%%
\addtocounter{framenumber}{-1}
%%%%%%%%%%%%%%%%%%%%%%%%%%%%%%%
\begin{frame}
 \frametitle{What is git? Example}
 
 \begin{center}
  \includegraphics[width=7cm]{figures/git_step_checkout.png}
 \end{center}
 \vspace*{-0.5cm}
 
 \begin{itemize}
  \color{gray}
  \item[] \texttt{vi test.dat}
  \item[] \texttt{git init}
  \item[] \texttt{git add test.dat}
  \item[] \texttt{git commit -m "Create my new file"}
  \item[] \texttt{rm test.dat}
  \color{black}
  \item   \texttt{git checkout test.dat}
  \color{white}
  \item[] \texttt{vi test.dat}
  \item[] \texttt{git add test.dat}
  \item[] \texttt{git commit -m "Update my file"}
  \item[] \texttt{git checkout [1] test.dat}
 \end{itemize}

\end{frame}
%%%%%%%%%%%%%%%%%%%%%%%%%%%%%%%
\addtocounter{framenumber}{-1}
%%%%%%%%%%%%%%%%%%%%%%%%%%%%%%%
\begin{frame}
 \frametitle{What is git? Example}
 
 \begin{center}
  \includegraphics[width=7cm]{figures/git_step_2.png}
 \end{center}
 \vspace*{-0.5cm}
 
 \begin{itemize}
  \color{gray}
  \item[] \texttt{vi test.dat}
  \item[] \texttt{git init}
  \item[] \texttt{git add test.dat}
  \item[] \texttt{git commit -m "Create my new file"}
  \item[] \texttt{rm test.dat}
  \item[] \texttt{git checkout test.dat}
  \color{black}
  \item   \texttt{vi test.dat}
  \color{white}
  \item[] \texttt{git add test.dat}
  \item[] \texttt{git commit -m "Update my file"}
  \item[] \texttt{git checkout [1] test.dat}
 \end{itemize}

\end{frame}
%%%%%%%%%%%%%%%%%%%%%%%%%%%%%%%
\addtocounter{framenumber}{-1}
%%%%%%%%%%%%%%%%%%%%%%%%%%%%%%%
\begin{frame}
 \frametitle{What is git? Example}
 
 \begin{center}
  \includegraphics[width=7cm]{figures/git_step_add2.png}
 \end{center}
 \vspace*{-0.5cm}
 
 \begin{itemize}
  \color{gray}
  \item[] \texttt{vi test.dat}
  \item[] \texttt{git init}
  \item[] \texttt{git add test.dat}
  \item[] \texttt{git commit -m "Create my new file"}
  \item[] \texttt{rm test.dat}
  \item[] \texttt{git checkout test.dat}
  \item[] \texttt{vi test.dat}
  \color{black}
  \item   \texttt{git add test.dat}
  \color{white}
  \item[] \texttt{git commit -m "Update my file"}
  \item[] \texttt{git checkout [1] test.dat}
 \end{itemize}

\end{frame}
%%%%%%%%%%%%%%%%%%%%%%%%%%%%%%%
\addtocounter{framenumber}{-1}
%%%%%%%%%%%%%%%%%%%%%%%%%%%%%%%
\begin{frame}
 \frametitle{What is git? Example}
 
 \begin{center}
  \includegraphics[width=7cm]{figures/git_step_commit2.png}
 \end{center}
 \vspace*{-0.5cm}
 
 \begin{itemize}
  \color{gray}
  \item[] \texttt{vi test.dat}
  \item[] \texttt{git init}
  \item[] \texttt{git add test.dat}
  \item[] \texttt{git commit -m "Create my new file"}
  \item[] \texttt{rm test.dat}
  \item[] \texttt{git checkout test.dat}
  \item[] \texttt{vi test.dat}
  \item[] \texttt{git add test.dat}
  \color{black}
  \item   \texttt{git commit -m "Update my file"}
  \color{white}
  \item[] \texttt{git checkout [1] test.dat}
 \end{itemize}

\end{frame}
%%%%%%%%%%%%%%%%%%%%%%%%%%%%%%%
\addtocounter{framenumber}{-1}
%%%%%%%%%%%%%%%%%%%%%%%%%%%%%%%
\begin{frame}
 \frametitle{What is git? Example}
 
 \begin{center}
  \includegraphics[width=7cm]{figures/git_step_checkout2.png}
 \end{center}
 \vspace*{-0.5cm}
 
 \begin{itemize}
  \color{gray}
  \item[] \texttt{vi test.dat}
  \item[] \texttt{git init}
  \item[] \texttt{git add test.dat}
  \item[] \texttt{git commit -m "Create my new file"}
  \item[] \texttt{rm test.dat}
  \item[] \texttt{git checkout test.dat}
  \item[] \texttt{vi test.dat}
  \item[] \texttt{git add test.dat}
  \item[] \texttt{git commit -m "Update my file"}
  \color{black}
  \item   \texttt{git checkout [1] test.dat}
  \color{white}
 \end{itemize}

\end{frame}

\begin{frame}[fragile]
 \frametitle{What is git? States of a file w.r.t. git}
 
 \texttt{git status -s}
 
 \begin{lstlisting}[style=Output]
%{\color{green}A\ }% stagged_first_time.dat
%{\color{green}M\ }% tracked_with_stagged_changes.dat 
%{\color{red}\ M}% tracked_with_unstagged_changes.dat 
%{\color{red}??}% untracked.dat
\end{lstlisting}
\vspace*{0.3cm}

Rest of files are tracked but not changed. 
\vspace*{0.2cm}

Or ignored with \texttt{.gitignore} file: list all untracked files that should be ignored by git
\vspace*{0.6cm}

Only tracked files are in the repository. Can be listed with \texttt{git ls-files} or using a gui (like gitk or gitg)
 
\end{frame}


\begin{frame}
 \frametitle{What is git? Whole scheme}
 
 Tools also exist to communicate with repositories in remote servers
  \vspace*{0.4cm}
 
 \begin{center}
 \includegraphics[width=6.5cm]{figures/move_on_git_world3.png}
 \end{center}
 
\end{frame}


\begin{frame}
 \frametitle{How does it works?}

\begin{itemize}
 \item All info about ``staging area'' and ``repository'' is stored in a folder on our working directory: {\color{blue}\texttt{.git/}}
 \vspace*{0.3cm}
 
 \item All tools are called in a similar way:
  \vspace*{0.1cm}

  \texttt{\$> {\color{blue}git toolname [options]}}
  \vspace*{0.2cm}
  
  \begin{itemize}
   \item Initiate a git repository on a folder (i.e., creates \texttt{.git/}):
   
   \texttt{git init}
   \vspace*{0.1cm}
   
   \item Add a file to the staging area:
   
   \texttt{git add filename}
   \vspace*{0.1cm}
   
   \item Write content from staging area to the repository:
   
   \texttt{git commit -m "Message explaining the changes"}
  
  \end{itemize}
  
\end{itemize}

\end{frame}

\begin{frame}
 \frametitle{Basic workflow}
 
 \textbf{Inserting git in our development protocol} \onslide<2->{\color{red}(with remotes)}
 \vspace*{0.5cm}
 
 \begin{enumerate}
  \item Make changes to the code  (\texttt{mycode.f90})
  \vspace*{0.2cm}
  
  \item Add them the the staging area: 
  
  \texttt{git add mycode.f90}
  \vspace*{0.2cm}
  
  \item Commit changes locally:
  
  \texttt{git commit -m "Bug X was fixed"}
  \vspace*{0.2cm}
  
  \pause
  
  \item Add them to the remote repository:
  
  \texttt{git push}
     \vspace*{0.2cm}
  
  \color{white}
  \item[] Before uploading, get current status of the remote repository:
  
  \texttt{git pull}
 \end{enumerate}

\end{frame}
%%%%%%%%%%%%%%%%%%%%%%%%%%%%%%%
\addtocounter{framenumber}{-1}
%%%%%%%%%%%%%%%%%%%%%%%%%%%%%%%
\begin{frame}
 \frametitle{Basic workflow}
 
 \textbf{Inserting git in our development protocol} \onslide<1->{\color{red}(with remotes)}
 \vspace*{0.5cm}
 
 \begin{enumerate}
  \item Make changes to the code (\texttt{mycode.f90})
  \vspace*{0.2cm}
  
  \item Add them the the staging area: 
  
  \texttt{git add mycode.f90} 
  \vspace*{0.2cm}
  
  \item Commit changes locally:
  
  \texttt{git commit -m "Bug X was fixed"}
  \vspace*{0.2cm}
  
  {
  \color{blue}
  \item[*] Before uploading, get current status of the remote repository:
  
  \texttt{git pull}
  }
  \vspace*{0.2cm}
  
%   \addtocounter{enumi}{-1}
  
  \item Add them to the remote repository:
  
  \texttt{git push}
 \end{enumerate}

\end{frame}



\begin{frame}
 \frametitle{History of the repository (with only 1 branch)}
 
 \begin{itemize}
  \item All objects (commits) in git have associated a unique ID (hash), e.g. \texttt{d6cd1e2bd19e03a81132a23b2025920577f84e37}
  \vspace*{0.2cm}
  
  \item The history can be examined with:

  \texttt{git log -{}-oneline -{}-decorate -{}-graph} (or with gui as \texttt{gitg}) 
  \begin{center}
   \includegraphics[width=11cm]{figures/log.png}
  \end{center}
    \vspace*{0.2cm}

   \item Changes in history should be avoided, even more in collaborative projects
  
 \end{itemize}

 
\end{frame}


\begin{frame}
 \frametitle{Travelling along history}
 
 \begin{itemize}
 
  \item Branch history line is linear (with some bifurcations if collaborative):
  
  \includegraphics[width=5.5cm]{figures/history.png}
  \vspace*{0.2cm}
  
  \item Many ways to come back to the past: workdir and/or repository(history).\\
  E.g., shifting to old commit d6cd1e2:
  \vspace*{0.2cm}
  
  \begin{minipage}{1.1\textwidth}
  \begin{itemize}
   \item Change workdir to commit \texttt{d6cd1e2} (but leave history untouched):\\
   \texttt{git checkout d6cd1e2}
   \vspace*{0.2cm}
   
   \item Shift history to commit \texttt{d6cd1e2}, (but leave workdir untouched):\\
   \texttt{git reset d6cd1e2}
   \vspace*{0.2cm}
   
   \item Reset history and workdir to commit \texttt{d6cd1e2} ({\color{red}\textbf{work is lost}}):\\
   \texttt{git reset d6cd1e2 -{}-hard}
  \end{itemize}
  \end{minipage}
  
 \end{itemize}

 
\end{frame}


\begin{frame}
\setbeamercovered{transparent}
 \frametitle{Configure local and remote servers}
 
 \begin{itemize}
  \item<1> Git identifies users by name and e-mail
  
  \texttt{git config -{}-global user.email "john@uni.edu"}
  
  \texttt{git config -{}-global user.name "John Doe"}
  \vspace*{0.2cm}

  \item<2> Starting a repository
  
  \texttt{git init} (on a local machine)
  
  \texttt{git init -{}-bare -{}-shared} (on a remote server)
  \vspace*{0.2cm}
  
  \item<3,6> Setting a remote server with protocols: local, ssh, http (no further config)
  
  \only<6->{\color{black!20!white}}
  \only<1-5>{\texttt{git remote add {\color{blue}local} file:///home/john/Repos/MyWork.git}}
  \only<6->{\texttt{git remote add local file:///home/john/Repos/MyWork.git}}
  
  \begin{minipage}{1.1\textwidth}
  \only<1-5>{\texttt{git remote add {\color{blue}cluster} ssh://john@cluster:\textasciitilde/Repos/MyWork.git}}
  \only<6->{\texttt{git remote add cluster ssh://john@cluster:\textasciitilde/Repos/MyWork.git}}
  \end{minipage}

  \alert<6>{
\texttt{git remote add {\color{blue}github} https://github.com/john/MyWork.git}}
  \vspace*{0.2cm}
  
  \item<4> Sending/getting commit history to/from remote
  
  \texttt{git push {\color{blue}cluster}} (send)
  
  \texttt{git pull {\color{blue}cluster}} (receive)
  \vspace*{0.2cm}
  
  \item<5> Clone from a remote repository
  
  \texttt{git clone john@cluster:\textasciitilde/Repos/MyWork.git}
  
 \end{itemize}

\end{frame}

\begin{frame}
 \frametitle{What is github?}
 
 ``GitHub is a web-based Git repository and Internet \textbf{hosting service}''
 \vspace*{0.8cm}
 
 \begin{itemize}
  \item Remote server with graphical interface
  \vspace*{0.3cm}
 
  \item Host unlimited number of open-source projects for free
   \vspace*{0.3cm}
   
  \item Additional features on top of git
  \vspace*{0.2cm}
  
  \begin{itemize}
   \item \textbf{Forks}: get a copy of a project on github into your account
   \vspace*{0.1cm}
   
   \item \textbf{Pull requests}: ask a developer to apply your changes into his project
  \end{itemize}
 \vspace*{0.5cm}
  
 \end{itemize}
 
 Other web services available: e.g. bitbucket
 \vspace*{0.2cm}
 
 Similar GUI can be mounted on a cluster: gitlab and gogs

 
\end{frame}

\begin{frame}
 \frametitle{Introduction to branches}
 
 \begin{itemize}  
  \item \textbf{Why using branches?}\\
  Allow to work independently on an specific feature of the project\\
  (useful, but depends on the development workflow)
 \end{itemize}
 
 \begin{itemize}
 \small
  \item Working with a remote we are actually using a branch (\texttt{remote/master})!
 \end{itemize}
 
 \only<1>{\hspace*{1cm}\includegraphics[width=6cm]{figures/branching_0.png}}
 \only<2>{\hspace*{1cm}\includegraphics[width=6cm]{figures/branching_1.png}}
 \only<3>{\hspace*{1cm}\includegraphics[width=6cm]{figures/branching_final.png}}
 \only<4>{\hspace*{1cm}\includegraphics[width=6cm]{figures/branching_wanamerge.png}}
 
 
 \begin{itemize}
  \item \only<2>{\texttt{git checkout -b Feature}}
        \only<3->{{\color{black!20!White}\texttt{git checkout -b Feature}}}
  \vspace*{0.1cm}
  
  \item<3-> \only<1-2>{.}\only<3>{(commits on both branches)}\only<4->{{\color{black!20!White}(commits on both branches)}}
  \vspace*{0.1cm}
  
  \item<4-> \textbf{Export work made on both branches}: \texttt{git merge} or \texttt{git rebase}
 \end{itemize}



 
 
\end{frame}

\begin{frame}
 \frametitle{Introduction to branches}

 
 \only<1-2>{\texttt{git merge Master} (from \texttt{Feature})}
 \only<3->{{\color{black!20!white}\texttt{git merge Master} (from \texttt{Feature})}}
 \vspace*{-0.6cm}
 
 \only<1>{\noindent\hspace*{1cm}\includegraphics[width=6cm]{figures/branchingB.png}}
 \only<2>{\noindent\hspace*{1cm}\includegraphics[width=6cm]{figures/branchingB_merge.png}}
 \only<3->{\noindent\hspace*{1cm}\includegraphics[width=6cm]{figures/branchingB_mergeFadeOut.png}}
 
 \vspace*{0.0cm}
 
 \pause
 \phantom{.}
 \pause
 
 \vspace*{-0.3cm}
 
 \only<1-4>{\texttt{git rebase Master} (from \texttt{Feature})}
 \only<5->{{\color{black!20!white}\texttt{git rebase Master} (from \texttt{Feature})}}
 \vspace*{-0.6cm}
 
 \only<1-3>{\noindent\hspace*{1cm}\includegraphics[width=6cm]{figures/branchingB.png}}
 \only<4>{\noindent\hspace*{1cm}\includegraphics[width=6cm]{figures/branchingB_rebase.png}}
 \only<5->{\noindent\hspace*{1cm}\includegraphics[width=6cm]{figures/branchingB_rebaseFadeOut.png}}

 \onslide<5->{Switch between branches: \texttt{git checkout <branchname>}}

 \end{frame}


%%%%%%%%%%%%%%%%%%%%%%%%%%%%%%%
\addtocounter{framenumber}{-1}
%%%%%%%%%%%%%%%%%%%%%%%%%%%%%%%
\begin{frame}[t,plain]
  \titlepage
\end{frame}


%%%%%%%%%%%%%%%%%%%%%%%%%%%%%%%%%%%%%%%%%%%%%%%%%%%%%%%%%%%%%%%%%%%%%
% BACK UP
%%%%%%%%%%%%%%%%%%%%%%%%%%%%%%%%%%%%%%%%%%%%%%%%%%%%%%%%%%%%%%%%%%%%%%

% %%%%%%%%%%%%%%%%%%%%%%%%%%%%%%%
% \addtocounter{framenumber}{-1}
% %%%%%%%%%%%%%%%%%%%%%%%%%%%%%%%
% 
% \begin{frame}[c]
%   \frametitle{Perspectives}
% %   \framesubtitle{Spectra simulation at Harmonic Approximation: vibrational structure}
% 
% 
% \end{frame}


\end{document}